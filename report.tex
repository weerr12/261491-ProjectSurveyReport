\documentclass[semifinal]{cpecmu}

\projectNo{S002-2/65}
\acadyear{2022}

\titleTH{เกมแนวสยองขวัญที่ใช้ความร่วมมือของผู้เล่นสองคน}
\titleEN{Two-player Cooperative Horror Game}

\author{นายธนดล เดชประภากร}{Tanadol Deachprapakorn}{630610734}
\author{นายภูริช สีนวลแล}{Purich Seenaullae}{630610752}

\cpeadvisor{karn}
\cpecommittee{sakgasit}
\cpecommittee{patiwet}

\usepackage[final]{graphicx}
\usepackage{subcaption}
\PassOptionsToPackage{hyphens}{url}
\usepackage{url}
\usepackage[colorlinks=true,allcolors=Blue4,citecolor=red,linktoc=all]{hyperref}
\def\UrlLeft#1\UrlRight{$#1$}
\usepackage{afterpage}
\usepackage{pdflscape}

\begin{document}

% \include{chapters/frontmatter}
\pagestyle{empty}\cleardoublepage
\normalspacing \setcounter{page}{1} \pagenumbering{arabic} \pagestyle{cpecmu}

% \include{chapters/intro}
% \include{chapters/background}
% \include{chapters/approach}
% \include{chapters/eval}

\ifproject
\include{chapters/conclusion}
\fi

\bibliography{report}
\bibliographystyle{plainurl}

\ifproject
\normalspacing
\appendix
\include{chapters/appendix}

\ifglossary\glossarypage\fi
\ifindex\indexpage\fi

\begin{biosketch}
  \begin{center}
    \includegraphics[width=1.5in]{./img/phu.png}
  \end{center}
  \textbf{นายธนดล เดชประภากร} Tanadol Deachprapakorn \\
  \textbf{รหัสนักศึกษา}: 630610734 \\
  \textbf{อีเมล}: \href{mailto:tanadol_de@cmu.ac.th}{tanadol\_de@cmu.ac.th}
  \begin{itemize}
    \item เข้าร่วม Global Game Jam 2023 ที่เชียงใหม่
    \item เข้าร่วม The 3rd Kibo Robot Programming Challenge 2022 จัดโดย สำนักงานพัฒนาวิทยาศาสตร์และเทคโนโลยีแห่งชาติ(สวทช.) และองค์การสำรวจอวกาศญี่ปุ่น (JAXA) - ได้รับรางวัล Top 20th Finalist เป็นผู้ชนะเลิศอันดับที่ 4
    \item เข้าร่วม Faipa Hackathon วิเคราะห์และบริหารวิกฤตการณ์ไฟป่าด้วยเทคโนโลยีดาวเทียมและ Big Data – ได้รับรางวัลรองชนะเลิศอันดับที่ 2 ร่วมคิดการบูรณการ Hotspot, sensor, การสื่อสาร วิธีการรับมือไฟป่า
    \item เข้าร่วมโครงการพัฒนาระบบนิเวศเพื่อสร้างผู้ประกอบการรุ่นใหม่ (Entrepreneurial Ecosystem Development) จัดโดย อุทยานวิทยาศาสตร์และเทคโนโลยี มหาวิทยาลัยเชียงใหม่
    \item เข้าร่วม สู้ (ทัน) ควัน 32hrs. Hackathon นวัตกรรม สู้หมอกควัน - ได้รับรางวัลชนะเลิศอันดับที่ 1 ร่วมคิดการตรวจจับไฟป่า และแจ้งเตือน โดยใช้อากาศยานไร้คนขับ
    \item เข้าร่วมโครงการเสวนาวิชาการและแข่งขัน Hackathon การมีส่วนร่วมของประชาชนในการแก้ไขรัฐธรรมนูญและเสนอนโยบาย ครั้งที่ 1 – ได้รับรางวัลชนะเลิศ